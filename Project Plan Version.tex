\documentclass{article}
\usepackage[utf8]{inputenc}
\usepackage{hyperref}

\title{Post-Doc Research Plan: \\\Large{Understanding and Evaluating Vaccine Safety Discussions and Expert Communication on Twitter}}
\author{Post-doctoral Researcher: David Manheim \\Faculty Advisor: Anat Gesser-Adelsburg}
\date{March 2019}

\begin{document}

\maketitle

\section{Introduction}
Risk Communication is a difficult but critical task in public health. The task of risk communication on social media is particularly challenging because of the rapidly changing nature of the platform, and the challenges of active engagement with the public. 

Twitter is being used extensively in various positive capacities in public health%\cite{a,b,c,d}
, but also serves as a location for misinformed or ill-intentioned individuals to spread misinformation%\cite{e,f,g,h}
. This is particularly true in the realm of disinformation and attempts to correct it regarding vaccine safety, and Twitter has therefore become one of several battlegrounds, and has been a source of disinformation and confusion%\cite{i,j,k}
, and risk communication for vaccines is made both more important, and more challenging.

While it seems that public health organizations are winning this battle, or at least are engaged in critical efforts to that end, success will require evaluation and adjustment of strategies. Despite recommendations about how best to use Twitter for health communication, and discussion about how these best practices are relevant to vaccine communication, not enough has been done to evaluate the use of Twitter by experts in this particular area. This type of evaluation is critical for assisting organizations in refining their strategies and further improving public health response. To support this, this research project hope to begin a process of better evaluating current efforts to inform the public and combat misinformation.

\section{Brief Review of Extant Literature}

General best practices for risk communication have been promulgated by the World Health Organization, including a course on risk communication essentials \footnote{\url{https://openwho.org/courses/risk-communication/items/650J3rv4o76fglMyYghSMY}}, and a guide specifically for vaccine safety has been produced by the Council for International Organizations of Medical Sciences (CIOMS) Working Group on Vaccine Safety\footnote{CIOMS Guide to vaccine safety communication. Report by topic group 3 of the CIOMS Working Group on Vaccine Safety. Geneva, Switzerland: Council for International Organizations of Medical Sciences (CIOMS), 2018.} 

There has been some work done on evaluation in related areas, but none has focused on health communication for vaccines and how the current efforts are effective. For instance, Messner et al. 2013 looked at the overall twitter usage of 20 nonprofit health organizations over a brief period in 2012, and looked at 948 tweets for use of hashtags, interaction with other twitter users, and other indications. Among other points, this paper is an early example that highlighted the importance of hashtags and user interaction in health risk communication.

Eckert et al. performed a literature review to understand best practices for risk communication around emergencies. The goal of their meta-analysis was to identify best platforms and practices to promote health protection and fight disinformation. Their conclusions included a consensus that social media should be used to communicate, but many studies noted that the communication needed to be contextualized properly. There was also consensus that  social media should be incorporated into daily operations of health education organizations. The reviewed studies also considered misinformation, and noted that in the context of disaster response, self-regulation by users (i.e. not experts) was often effective\footnote{This stands in contrast to vaccination debates, where wisdom of the crowds seems to have failed to quash misinformation. This is likely due to non-expert's ability to personally verify misinformation during crises.}. Despite this, the studies recommended experts and partners assign dedicated social media officers who build trust and credibility, and are able to rapidly respond. The studies also found that hashtags, especially organic ones, were useful for communication, especially for "prompt myth-busting."\footnote{Again, the difference between crisis response, where hashtags are used extensively by people looking for information in real time, differs from how it is used for endemic and ongoing issues like vaccine denial. On the other hand, there are theoretical results showing that this type of fake news should be identifiable by simple agents. \cite{Aymanns2017}.} Importantly, Eckert et al. noted that "studies on social media use during disasters were published demanding a new evaluation of emerging evidence on best practices for governmental agencies, implementing partners and the public." \cite{Eckert2018}

Elbanna et al. more recently looked at operational implications of social media for emergency management, and proposed a research agenda on the basis of workshops with individuals and organizations. This paper identified six challenges, "which include the dissemination of information, disaster tourism, communities’ expectations, flash volunteering, PPRR, crisis communication and community relations." \cite{Elbanna2019} Some of these issues seem likely to emerge in our investigation as well, and the relationship between the identified challenges and what we expect to find will be explored.

\section{Research Goals}

Unfortunately absent from the literature were attempts to evaluate the identified best practices, and this is the key area that the current work hopes to begin to address. The significant gap in the literature on evaluating social media engagement in public health means that there is little feedback about how well organizations are faring in implementing the suggested best practices. Vaccine risk communication is both itself an important topic, and a useful case-study for how to understand and evaluate other risk communication efforts.

The primary goal of the study is to identify how organizations and individuals involved in vaccine safety risk communication are acting in practice, and to compare this to the promulgated best practices. The case study will analyze twitter accounts involved in relevant communications, collect data on their interactions, and analyzing the data to understand whether and how well the existing recommendations and best practices are followed. 

In order to ensure the analysis is valid, the research procedure is defined in this project plan, before the data is collected. This will significantly limit researcher degrees of freedom, and hopefully allow peer review of the procedure without reference to the results. %\textit{TENTATIVE: To this end, open peer review of the protocol is being sought.}

\subsection{Publications and Targeted Journals}
The primary publication from the research project is a paper evaluating health expert communications. This paper may include evaluation or comparison of pro- and anti-vaccine advocates, or a secondary paper with this type of comparison may be written. There may also be a more health policy oriented op-ed or article presenting conclusions and recommendations.

Depending on the results of the study, the work may be more or less impactful. If particularly noteworthy results are found, this would suggest submitting to a more prestigious journal. A tentative list of journals that the papers might be submitted to, in rough order of preference, include:
\begin{itemize}
\item BMJ
\item PNAS
\item Risk Analysis 
\item Journal of Risk Research
\item PLoS ONE
\item Implementation Science
\item Journal of Medical Internet Research
\end{itemize}


\section{Risk Communication Recommendations}

There are a variety of recommendation types, some of which will be more amenable to analysis than others.

\subsection{Initial Literature Review}
The World Health Organization (WHO) provides an online course via their OpenWHO platform entitled "Risk communication essentials," available online (\href{https://openwho.org/courses/risk-communication}{LINK})\cite{OpenWHORiskComm2017}. Some of the points raised in this training are possible to evaluate, while others are not. 
 
For example, having a "Single Overarching Communication Outcome" is both hard for an external party to evaluate, and requires inferences about specific organizations' decisions and goals. Other recommended tactics, such as acknowledging the question and transitioning back to the key message, are evaluate-able but may be less useful or relevant in a length-restricted tweet. Fro example, recommendations such as ensuring messages start with why, then discuss how, and finally address what, are less relevant when only 280 characters are available.

Some points are easier to evaluate. The WHO training recommends that messages should "avoid jargon and technical language," that they "acknowledge uncertainty[,] mistakes [and] what you don't know," and they are respectful of concerns." Even easier to evaluate are specific rules, that messages should "avoid negative terms (not, never, no...)[, and] absolute terms (always, never, absolutely certain...)"

The WHO itself engages in outreach on social media, as they discuss on their web site\cite{WHOWebSocialMedia}, and they describe their strategy. "By participating in social media conversations and disseminating credible information through social media, WHO can drive traffic to the WHO website where more detailed and trusted content can be found." Referencing trusted sources and linking to that content is both a viable strategy on twitter, and one that can be evaluated.

%\subsection{DRAFT / NOTES SECTION re: WHO and guidelines}
%{ \textit{WHO also recommends using strategies from "Communication for Behavioral Impact (COMBI)" in the course - I still need to find and review this.} }

%{ \textit{WHO also provides various guidelines on their communication.  \url{https://www.who.int/communicating-for-health/principles/en/} }

%\textit{In a 2011 article, they say \url{https://www.who.int/bulletin/volumes/89/11/11-031111/en/};
%Social media and the web give people a forum to express their views, but these are debates that may never reach a conclusion. For public health specialists this can be difficult ground. “Once you start interacting with social media, you have to be ready to carry the work forward”, says Elbes of WHO. “You can't just start and stop.”}
%}

\subsection{Understanding the Corpus}

Based on the WHO Risk Communications essentials course, we consider four categories of people; Champions, Silent Boosters, Avoiders, and Blockers. In the training, the WHO provides a "Quadrant Diagram" \cite{Rao2009b} where the two specta represented by the axes are Agree/Disagree and Energetic/Passive. In our case, this is Pro-vaccine, Anti-vaccine, and a spectrum of how engaged the people are on twitter. As Rao notes, this is useful for representing areas where there is ambiguity or fuzziness in (at least one of) the categories, but where despite the inability to specify an exact taxonomy, the four quadrants are different paradigms, with clear examples. The four paradigmatic examples in our analysis respectively correspond to: experts and those who promote vaccines, the non-vocal vaccination-using public, those who are unsure whether to vaccinate or quietly choose not to vaccinate, and vocal doubters and anti-vaccine activists. The primary accounts we expect to find in our corpus are the first and last group, but it is critical to notice that the most important audience is likely to be the (unobserved) third group which sees the twitter conversations\footnote{Some of these accounts might be able to be identified via likes or retweets, or at their most vocal, questions and ambivalent statements in these conversations, but these are not the focus of the corpus or the evaluations.}

\section{Research Plan}

The research plan is tentative, and will be a focus of the initial collaboration. The current plan, including data sources and the analysis plan, is outlined below.

\subsection{Data Sources and Data Collection}
The data for the current study will primarily consist of a corpus of tweets retrieved from Twitter. The process for defining and retrieving this corpus is somewhat complex, and decisions and procedures for data collection are critical to the validity and relevance of a study. For this reason, the procedures are outlined, and the code for retrieval and analysis will be made available along with the analysis. To the extent possible, the dataset will be made available as well\footnote{Twitter policies make publicly providing the full dataset problematic, but tweet IDs can be published for others to examine.}.

To outline the procedure, first, a list of organizations that are involved in vaccine health education will be compiled, along with relevant twitter accounts.  Tweets from the identified accounts that mention terms used in discussion of vaccine safety and concerns about vaccine safety will be collected, along with the conversations, as described below. Depending on the cost and logistical difficulty, the initial list of accounts may be extended to both additional health education accounts, and to anti-vaccine propaganda accounts, using the procedures described below.

\subsection{Twitter Data and Barriers to Access}
Twitter unfortunately restricts access to its data for academic purposes. \cite{Alaimo2018} Full access to the described dataset will be requested, and if it is possible within a reasonable budget, this will be used. Otherwise, Twitter API provides a way to query twitter and retrieve a sample of data, which can then be used to find relevant users and accounts as well as replies and discussions. 

Note that the described data collection is permitted under Twitter's policies. Tracking these groups and individuals should be allowed given that neither anti-vaccine groups nor health experts should be considered "sensitive groups and organizations." Any links between twitter and off-twitter data will be solely on the basis of claimed affiliations, and will be used only to identify whether someone is a health expert at a university, a national or international organization, or a government agency. The data will also not be used to "derive or infer potentially sensitive characteristics".

\subsubsection{Account Lists}

The World Health Organization has a "Vaccine Safety Network", with twitter account @WHO\_VSN. This account follows a set of different account that appear to be members of this network, and these twitter accounts will form the primary list of vaccine health educators.  This includes both organizational accounts, individuals who represent organizations, and the accounts of other recognized experts who are involved in risk communication efforts and providing information about the importance of vaccination and vaccine safety.

The set of accounts that these accounts follow will be extracted to potentially supplement this list\footnote{A small selection of these accounts have been briefly inspected to ensure the reasonableness of the procedure.}. This is likely to be too extensive a list, so accounts that are "verified," are followed by multiple of these primary accounts, or are followed by a large number of verified accounts will be prioritized. Not all of these additional accounts will be relevant to vaccine risk communication, so these will be filtered. Accounts will be included if their account names or description contains terms relevant to vaccines or public health. The terms indicating account relevance are: Health, Doctor(s), Medicine, Medical, Vaccine(s), Infectious Disease(s), Immunize, and Immunization.

Depending on the size of the corpus of tweets obtained, the primary corpus may also be used to identify additional key accounts, either those that are frequently involved in the conversations, those that frequently retweet the conversations, or those that belong to celebrities or are important due to having very large numbers of followers.

In addition to these accounts, accounts involved in anti-vaccine propaganda may be examined in order to both better understand interactions, and to consider how those accounts engage with followers and other twitter users, and how they promote misinformation. This would be done in the same way the primary health education corpus is used to identify additional key accounts. Specifically, this will identify the most important twitter users that reply to discussions about vaccines and include anti-vaccine terms, as discussed below.

\subsubsection{Tweet and Discussion Corpus Extraction}

For each health education account, the list of tweets sent \textit{(TENTATIVELY) between 1/1/2018 and 1/1/2019} will be extracted. Any tweet which mentions any of the vaccine-relevant keywords or hashtags listed below will be considered a relevant tweet for the corpus. The discussions involving these tweets will then be extracted. (For example, if a tweet from an expert replies to another tweet stating that, "MMR vaccines do not contain mercury"\footnote{They do not and never have done so\url{https://www.cdc.gov/vaccinesafety/concerns/thimerosal/index.html}.}, we would retrieve not only replies to the expert's tweet, and further tweets in the thread, but also the tweet to which it replied, along with replies and tweets in that conversation.)

The health education tweet term list is; Vaccine(s), Immunocompromised, Preventable disease(s), Rumors, Debunk(ed/ing), Expert(s).

In these discussions, we expect to find some amount of debate and anti-vaccine propaganda. The tweet term list for identifying misinformation and vaccine denial includes; Autism, Poison(s), Shills, Pharma, Mandatory, VAERS\footnote{The rare and typically minor side effects of vaccines that are collected by the Vaccine Adverse Event Reporting System (VAERS) in the United States is often cited by misinformed anti-vaccine advocates as proof that vaccines are harmful.}, Aluminum, Mercury, and Thimerosal.

The accounts involved in questioning safety and promoting anti-vaccine misinformation will be identified in part based on this term list, which may be modified based on review of the data, using the proposed analytical model which will be developed, partially discussed below.

\subsubsection{Code for Extraction and Analysis of Data}
The programming for retrieval and analysis of the data will be written as part of the project, and will be hosted in a public repository, with the exclusion of the cryptographic tokens and other private information needed to access the data.

\subsection{Metrics, Analysis Plans, and Tentative Hypotheses}

The primary metrics used to assess accounts will be determined more fully in the course of the research.

We expect that tweets in the corpus of conversations will be polarized due to the by-default combative nature of twitter interactions, and the increasing polarization of the debate. We expect that pro- and anti- vaccine participants in the discussion will differ in their use of the key terms mentioned above. Specifically, Pro-vaccine accounts will mention the identified health education terms first in conversations, while anti-vaccine accounts will mention the identified misinformation terms first in discussion. (This hypothesis will be checked based on whether it applies in a manual classification of a subset of accounts.)

It may be difficult to automatically categorize the participants into pro-and anti-vaccine with high accuracy based purely on terms used, but manual tagging of a selected subset can allow machine learning classification models for to be used. From this subset, features will be extracted. Features are likely to include links to known (pro- or anti- vaccine) domains, and terms from the earlier term list.

The current plan is that Multinomial Naive Bayes will be used based on the features list on a per-user basis to classify them into pro- or anti-vaccine. If these achieve greater than 90\% accuracy on a validation set, the results of this will be used. Otherwise, manual classification will be used for (a subset of) the corpus (if very large,) of conversations to provide (indicative) results.

\bibliographystyle{alpha}
\bibliography{Postdoc}

\end{document}
